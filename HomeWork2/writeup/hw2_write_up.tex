
% ===============================================
% CS444: HW2 I/O Elevators         Fall 2017
% hw_revised.tex
% Template for revised homework submission
% ===============================================
%         READ THE FOLLOWING CAREFULLY!!!
% ===============================================
% When you produce a PDF version of this document
% to turn in, change the filename to hwX-name.pdf
% replacing X with the homework assignment number
% and name with your last name.
% ===============================================


% -----------------------------------------------
% The preamble that follows can be ignored. Go on
% down to the section that says "START HERE"
% -----------------------------------------------

\documentclass{article}

\usepackage[margin=1in]{geometry}
\usepackage{amsmath,amsthm,amssymb}

\newcommand{\R}{\mathbb{R}}
\newcommand{\Z}{\mathbb{Z}}
\newcommand{\N}{\mathbb{N}}
\newcommand{\Q}{\mathbb{Q}}
\newcommand{\C}{\mathbb{C}}

\newenvironment{theorem}[2][Theorem]{\begin{trivlist}
\item[\hskip \labelsep {\bfseries #1}\hskip \labelsep {\bfseries #2.}]}{\end{trivlist}}
\newenvironment{lemma}[2][Lemma]{\begin{trivlist}
\item[\hskip \labelsep {\bfseries #1}\hskip \labelsep {\bfseries #2.}]}{\end{trivlist}}
\newenvironment{exercise}[2][Exercise]{\begin{trivlist}
\item[\hskip \labelsep {\bfseries #1}\hskip \labelsep {\bfseries #2.}]}{\end{trivlist}}
\newenvironment{problem}[2][Problem]{\begin{trivlist}
\item[\hskip \labelsep {\bfseries #1}\hskip \labelsep {\bfseries #2.}]}{\end{trivlist}}
\newenvironment{question}[2][Question]{\begin{trivlist}
\item[\hskip \labelsep {\bfseries #1}\hskip \labelsep {\bfseries #2.}]}{\end{trivlist}}
\newenvironment{corollary}[2][Corollary]{\begin{trivlist}
\item[\hskip \labelsep {\bfseries #1}\hskip \labelsep {\bfseries #2.}]}{\end{trivlist}}

\newenvironment{solution}{\begin{proof}[Solution]}{\end{proof}}

\begin{document}

% ------------------------------------------ %
%                 START HERE                 %
% ------------------------------------------ %

\title{HW2 Report} % Replace X with the appropriate number
\author{Bryce Egley, Bruce Garcia\\CS 444} % Replace "Author's Name" with your name

\maketitle
\hrule


% -----------------------------------------------------
% The following two environments (theorem, proof) are
% where you will enter the statement and proof of your
% first problem for this assignment.
%
% In the theorem environment, you can replace the word
% "theorem" in the \begin and \end commands with
% "exercise", "problem", "lemma", etc., depending on
% what you are submitting. Replace the "x.yz" with the
% appropriate number for your problem.
%
% If your problem does not involve a formal proof, you
% can change the word "proof" in the \begin and \end
% commands with "solution".
% -----------------------------------------------------

% ---------------------------------------------------------
% Design Plan
%----------------------------------------------------------
\section*{Design Plan}
Our Design Plan
\newline
The Clock algorithm is a disk scheduling algorithm. This works with a circular linked list which we first used in Data Structures. This determines the order for read and write requests. If there aren’t any request in the head direction then the algorithm will reverse and move the opposite way. To implement this we first checked if the queue was empty and added a new request if it was. Then we moved through the queue and compared values. We then would add the new current value to the tail when we found the one next in order.

% ---------------------------------------------------------
% Questions
%----------------------------------------------------------
\section*{Questions}

\begin{question}{1}
What do you think the main point of this assignment is?
\end{question}
TODO

% -----------------------------------------------------
% Second problem
% -----------------------------------------------------

\vspace{0.25in} % This just adds some space between problems 1 and 2.

\begin{question}{2}
How did you personally approach the problem? Design decisions, algorithm, etc.\end{question}
TODO

% -----------------------------------------------------
% Third problem
% -----------------------------------------------------

\vspace{0.25in} % This adds some space between problems 2 and 3.

\begin{question}{3}
How did you ensure your solution was correct? Testing details, for instance.
\end{question}
TODO

% ---------------------------------------------------------
% Fourth problem
%----------------------------------------------------------

\vspace{0.25in}
\begin{question}{4}
What did you learn?
\end{question}
TODO

% ---------------------------------------------------------
% Fifth problem
%----------------------------------------------------------

\vspace{0.25in}
\begin{question}{4}
How should the TA evaluate your work? Provide detailed steps to prove correctness.
\end{question}
TODO

% ---------------------------------------------------------
% Version Control Log
%----------------------------------------------------------
\section*{Version Control Log}
\textit{Table of Version Control Log} Ran command 'git log --graph --oneline'
\newline
\begin{center}
\begin{tabular}{ c c }
* 1bceaac Finished add request \\
* a72882d Cycle though queue \\
* b84eb71 added request func \\
* 6bf0183 Removed duplicate init method \\
* 48979c0 Print Kernel changes \\
* 785f98e changed var names \\
* be170ff modifed noop to sstf \\
* 044ab88 Create HW2 Directory \\
* 8932df9 Decrement numbers on phil \\
* 06f2a68 Fixed print \\
* 0b62556 Fixed lines print \\
* e7d70f5 Fixed display names \\
* dbfbc85 Fixed print names \\
* fb9697f Fixed print \\
* 972647d Fixed compile errors \\
* 90c8bd4 Removed fork structure \\
* 5efda41 delete philosopher name \\
* fca30a7 Fixed print \\
* 7eb6578 fixed comp errors \\
* af6c168 Fixed wait error \\
* 751afa1 fixed compling errors \\
* c659d3f created Makefile
\end{tabular}
\end{center}

% ---------------------------------------------------------
% Work Log
%----------------------------------------------------------
\section*{Work Log}
October 29, 2017: (2 hours) We read and analyzed assignment 2. And planned out how we would complete the look algorithm.
\newline
October 30, 2017: (3 hours) We went to the block folder in linux yocto and looked over the noop-iosched.c file. We renamed it to the assignment name ssftp.c
Then we tried to figure out what it did.
We looked up how to use the Kernel Print to use the program we would write. We added kernel print to all the non-struct methods.
\newline
October 30, 2017: (4 hours) We implemented the add request function for the look algorithm.
% -----------------------------------------------
% Ignore everything that appears below this.
% -----------------------------------------------

\end{document}
